
\documentclass{aastex63}

\newcommand{\be}{\begin{eqnarray}}
\newcommand{\ee}{\end{eqnarray}}
\def\la{\mathbin{\lower 3pt\hbox
      {$\rlap{\raise 5pt\hbox{$\char'074$}}\mathchar"7218$}}}
\def\ga{\mathbin{\lower 3pt\hbox
      {$\rlap{\raise 5pt\hbox{$\char'076$}}\mathchar"7218$}}} %> or of order
\renewcommand{\vec}[1]{\mathbf{#1}}
%\renewcommand{\vec}[1]{\ensuremath{\boldsymbol{#1}}} %boldface vector style
\newcommand{\grad}{\mathbf{\nabla}}

\received{}
\revised{}
\accepted{\today}
\submitjournal{APJ}


\shorttitle{Lyman $\alpha$}
\shortauthors{Arras et al.}

\graphicspath{{./}{figures/}}


\begin{document}

\title{Resonant Scattering in a Uniform Sphere with Large Optical Depth}



\correspondingauthor{Phil Arras}
\email{arras@virginia.edu}

\author{student}
\author{Phil Arras}
\author{Shane Davis}
\affiliation{Department of Astronomy, University of Virginia, Charlottesville, VA 22904, USA}


\begin{abstract}

The solution of the radiative transfer equation for resonant scattering of Lyman $\alpha$ photons for a uniform sphere of constant gas density is found in the limit of large line center optical depth. A monochromatic source of photons is assumed at the center of the sphere, and the intensity within the sphere and at the surface is found with the Eddington approximation for the angular dependence. The solution can be represented a sum of three terms: a simple, analytic solution of the inhomogeneous equation which diverges at the center of the sphere and at the emission frequency; a semi-analytic solution of the homogeneous equation which is required for the mean intensity $J=0$ at the surface; and finally, a solution of the homogeneous equation which enforces the correct, zero ingoing intensity boundary conditions at the surface. It is the compact, analytic form as well as the piece accounting for the correct, frequency dependent boundary condition that is novel in this study. The analytic solution is compared to numerical solutions with the Monte-Carlo method, which are valid at arbitrary optical depth, and the deviations from exact solution are investigated.


\end{abstract}


\keywords{}



\section{Introduction} \label{sec:intro}

Nice summary paragraph from recent Bourier  papers.

Hubble Space Telescope observations have found large Lyman $\alpha$ transit depths around the gas giants HD 209458b \citep{2003Natur.422..143V}, HD 189733b \citep{2012A&A...543L...4L} and 55 Cnc b \citep{2012A&A...547A..18E}, Neptune-size planets GJ 436 b 
\citep{2015Natur.522..459E}
%, 2017A&A...605L...7L,2019A&A...629A..47D}   
and GJ 3470 b \citep{2018A&A...620A.147B}, and possibly the super-Earth HD 219134b \citep{2019EPSC...13.1928L}.
Non-detections or marginal detections have been reported for the super-Earths 55 Cnc e \citep{2012A&A...547A..18E}, super-Earth HD 97658 b \citep{2017A&A...597A..26B} , super-Earth GJ 1132 b \citep{2019AJ....158...50W}, super-Earth $\pi$ Men c \citep{2020ApJ...888L..21G}, and marginal detections were  Earth-size TRAPPIST-1 \citep{2017A&A...599L...3B} and sub-Earth Kepler-444 \citep{2017A&A...602A.106B}.


\section{ previous analytic work }

\citet{1973MNRAS.162...43H, 1974MNRAS.166..373H} started work on analytic solutions as you get a Laplace equation. Doesn't get boundary condition right.
\citet{1990ApJ...350..216N} has analytic solutions following Harrington.
\citet{1990ApJ...350..216N} improved on Harrington. Does have divergent inhomogeneous solution in slab geometry! Also has discussion of absorption.
\citet{2006ApJ...649...14D} generalized Harrington to spherical geometry
\citet{1976ApJ...208..286W} tries to compute the radiation force
\citet{1994ApJ...427..603R} discuss an ansatz for the fokker-planck approximation where the full voigt profile is used.
\citet{2020arXiv200509692L} discusses solutions for the sphere. Follows \citet{2006ApJ...649...14D}. They cite \citet{2015MNRAS.449.4336S} on core skipping.
\citet{2002ApJ...567..922A,2015MNRAS.449.4336S} discusses acceleration.
\citet{2015MNRAS.449.4336S}: amounts to drawing the atom velocity from a truncated Gaussian distribution to force it back into the wing.  The logic seems to be that almost all of the core scattering results in zero spatial diffusion until you get far enough from the line wing.  It seems like this is just calibrated by modifying the truncation until the spectrum doesn’t seem to depend sensitively on the choice of truncation.  


%\begin{figure}
%\plotone{KT_Eri.pdf}
%\caption{The Swift/XRT X-ray light curve for the first year after
%outburst of the suspected recurrent nova KT Eri. At a maximum count rate of 
%328 ct/s, KT Eri was the brightest nova in X-rays observed to date. All 
%the component figures (6) are available in the Figure Set. Note that
%these components that are {\bf not} shown in the compiled pdf. The figure
%set consists of the same figures as shown in Figure \ref{fig:pyramid}. 
%The example figure shown for figure sets can be one component or many. 
%\label{fig:fig4}}
%\end{figure}



\acknowledgments




\appendix

\section{ derivation of the transfer equation }

\citet{1973MNRAS.162...43H} first showed that the transfer equation for the mean  intensity $J$ will satisfy a Poisson equation involving second derivatives of space and frequency variables. In this section we will briefly review the derivation of this equation including photon destruction terms and an emission term.

The radiative intensity $I = dE/(dA dt d\Omega d\nu)$ is the energy per perpendicular area $dA$, per time $dt$, per solid angle $d\Omega$ and per frequency $d\nu$ \citep{1986rpa..book.....R}. Here the time-dependence of $I$ has been ignored, which assumes the fluid is changing on timescales long compared to a light-crossing time. The intensity $I=I(\vec{x},\vec{n}, \nu)$ will be considered a function of position $\vec{x}$, photon (unit) direction vector $\vec{n}$, and cyclic frequency $\nu$. In the Eddington and two-stream approximations, $I(\vec{x},\nu) \simeq J(\vec{x},\nu) + 3 \vec{n} \cdot \vec{H}(\vec{x},\nu)$, where $J=(1/4\pi) \int d\Omega I$ is the mean intensity and $\vec{F} = 4\pi \vec{H}= \int d\Omega \vec{n} I$ is the flux.  

Photons of frequency $\nu$ near line center frequency $\nu_0$ are considered. The Doppler width will be written $\Delta = \nu_0 v_{\rm th}/c$, where $v_{\rm th}=\sqrt{2k_{\rm B}T/m_{\rm H}}$ is the thermal speed of hydrogen atoms of mass $m_{\rm H}$ and temperature $T$, and $c$ is the speed of light. The photon frequency in Doppler units will be written $x = (\nu-\nu_0)/\Delta$. For upper-state de-excitation rate $\Gamma$, the ratio of natural to Doppler broadening is $a=\Gamma/(4\pi \Delta)$. 
The transfer equation is \citep{1986rpa..book.....R}
\be
\vec{n} \cdot \grad I & =& - \left( \alpha_{\rm sc} + \alpha_{\rm abs} \right) I + (1-p) j_{\rm sc} + j_{\rm em}.
\label{eq:rteqn}
\ee
The scattering coefficient, or inverse mean free path to scattering, is 
\be
\alpha_{\rm sc} & = & n_{\rm sc}\, \frac{\pi e^2}{m_e c}\, f\, \frac{H(x,a)}{\sqrt{\pi} \Delta}
= k \phi   
\ee
where $n_{\rm sc}$ is the number density of scatterers, $e$ and $m_e$ have their usual meaning, $f$ is the oscillator strength of the transition, $H(x,a)$ is the Voigt function, $k = n_{\rm sc} (\pi e^2/m_e c) f$, and the Voigt line profile is $\phi = H(x,a)/(\sqrt{\pi} \Delta)$, which is normalized as $\int d\nu\, \phi(\nu) = 1$. The absorption coefficient $\alpha_{\rm abs}$, or inverse mean free path to true absorption, is a sum over number density of the absorber times absorption cross section. Once the incoming photon has promoted the electron to an excited state, the collisional de-excitation probability is $p$, and hence only a fraction $1-p$ of the excitations lead to re-emission of photons. ``Hummer Case II-b"  \citep{1962MNRAS.125...21H} will be used for the redistribution function, for which the incoming photon is absorbed by the atom according to the natural broadening in the rest frame, re-emitted with a dipole phase function $g(\vec{n},\vec{n}^\prime)=(3/16\pi)(1+[\vec{n}\cdot \vec{n}^\prime]^2)$, which is appropriate for a 1s-2p transition \citep{1982qe}, and the result is averaged over a Maxwell-Boltzmann distribution of speeds for the atom. The result can be written
\be
j_{\rm sc}(\vec{x},\vec{n},\nu) & = & k \int \frac{ d^3v}{ \pi^{3/2} v_{\rm th}^3} e^{-v^2/v_{\rm th}^2}\, 
\int d\Omega^\prime \int d\nu^\prime \,
g(\vec{n},\vec{n}^\prime) 
\nonumber \\ & \times & 
\delta \left( \nu - \nu^\prime - \nu_0 \vec{v} \cdot (\vec{n}-\vec{n}^\prime)/c \right)
\left( \frac{\Gamma/4\pi^2}{ \left(\nu^\prime - \nu_0 - \nu_0 \vec{v} \cdot \vec{n}^\prime/c \right)^2 + (\Gamma/4\pi)^2 } \right)  \,
I(\vec{x},\vec{n}^\prime,\nu^\prime)
\nonumber \\ & = & 4\pi k \int d\Omega^\prime \int d\nu^\prime R(\vec{n},\nu; \vec{n}^\prime,\nu^\prime) I(\vec{x},\vec{n}^\prime,\nu^\prime),
\ee
which defines the redistribution function
\be
R(\vec{n},\nu; \vec{n}^\prime,\nu^\prime) & = & \frac{ g(\vec{n},\vec{n}^\prime) }{ 4\pi }
\int \frac{ d^3v}{ \pi^{3/2} v_{\rm th}^3} e^{-v^2/v_{\rm th}^2}\,
\delta \left( \nu - \nu^\prime - \nu_0 \vec{v} \cdot (\vec{n}-\vec{n}^\prime)/c \right)
\left( \frac{\Gamma/4\pi^2}{ \left(\nu^\prime - \nu_0 - \nu_0 \vec{v} \cdot \vec{n}^\prime/c \right)^2 + (\Gamma/4\pi)^2 } \right).
\ee 
Our definition of $R$ differs from that of \citet{1962MNRAS.125...21H} by a factor of $4\pi$.

The integral of the redistribution function over outgoing and incoming frequency are
\be
\int d\nu\ R(\vec{n},\nu; \vec{n}^\prime,\nu^\prime) 
& = & \frac{1}{4\pi} g(\vec{n},\vec{n}^\prime) \phi(\nu^\prime)
\ee 
and
\be
\int d\nu^\prime \ R(\vec{n},\nu; \vec{n}^\prime,\nu^\prime) 
& = & \frac{1}{4\pi} g(\vec{n},\vec{n}^\prime) \phi(\nu)
\ee 
where the right hand side is the usual Voigt function, the thermal average of the Lorentzian. The former result implies that the integrated source and sink terms for scattering cancel for $p=1$. In addition, $d\nu d\Omega 4\pi R(\vec{n},\nu; \vec{n}^\prime,\nu^\prime)/\phi(\nu^\prime) $ is the normalized distribution for the outgoing $\vec{n}$ and $\nu$ given the incoming $\vec{n}^\prime$ and $\nu^\prime$. 

This probability distribution can be used to define the moments of the frequency shift
\be
\langle \Delta \nu^n \rangle & = & \frac{ \int d\nu^\prime R (\nu-\nu^\prime)^n}{\int d\nu^\prime R}
= \frac{1}{\phi(\nu)}
\int \frac{ d^3v}{ \pi^{3/2} v_{\rm th}^3} e^{-v^2/v_{\rm th}^2}\,
\left( \frac{\nu_0 \vec{v} \cdot (\vec{n}-\vec{n}^\prime) }{c} \right)^n
\left( \frac{\Gamma/4\pi^2}{ \left(\nu - \nu_0 - \nu_0 \vec{v} \cdot \vec{n}/c \right)^2 + (\Gamma/4\pi)^2 } \right),
\ee
which are functions of $\nu$, $\vec{n}$ and $\vec{n}^\prime$. These integrals can be evaluated in terms of the dimensionless moments of the parallel velocity distribution, defined as
\be
\langle u_\parallel^n \rangle(x) & = & \frac{a/\pi }{H(x,a)} \int 
\frac{du_\parallel u_\parallel^n e^{-u_\parallel^2}  }{(x-u_\parallel)^2 + a^2}.
\ee
The end results for the first and second moments are
\be
\langle \Delta \nu \rangle & = & \Delta \langle u_\parallel \rangle \left( 1 - \vec{n} \cdot \vec{n}^\prime \right)
\\
\langle \Delta \nu^2 \rangle & = & \Delta^2 
\left[ \langle u_\parallel^2 \rangle
\left( 1 - \vec{n} \cdot \vec{n}^\prime \right)^2
+ \frac{1}{2} \left( 1 - \vec{n} \cdot \vec{n}^\prime \right)^2 \right].
\ee

For small frequency shifts $\nu-\nu^\prime$, the incoming intensity may be expanded as 
\be
I(\vec{x},\vec{n}^\prime,\nu^\prime) & \simeq  &
I(\vec{x},\vec{n}^\prime,\nu) 
+ 
\frac{\partial I(\vec{x},\vec{n}^\prime,\nu)}{\partial \nu }(\nu^\prime-\nu)
+ \frac{1}{2} \frac{\partial^2 I(\vec{x},\vec{n}^\prime,\nu)}{\partial \nu^2} (\nu^\prime-\nu)^2 + ...
\ee 
and the Fokker-Planck expansion of $j_{\rm sc}$ is
\be
j_{\rm sc}(\vec{x},\vec{n},\nu) & \simeq  & 4\pi k \int d\Omega^\prime \int d\nu^\prime R(\vec{n},\nu; \vec{n}^\prime,\nu^\prime) 
\left[ 
I(\vec{x},\vec{n}^\prime,\nu) + \frac{\partial I(\vec{x},\vec{n}^\prime,\nu)}{\partial \nu }(\nu^\prime-\nu)
+ \frac{1}{2} \frac{\partial^2 I(\vec{x},\vec{n}^\prime,\nu)}{\partial \nu^2 }(\nu^\prime-\nu)^2
\right]
\nonumber \\ 
& =& 
k\phi(\nu) \int d\Omega^\prime g 
\left[ 
I(\vec{x},\vec{n}^\prime,\nu) 
- \frac{\partial I(\vec{x},\vec{n}^\prime,\nu)}{\partial \nu }
\langle \Delta \nu \rangle
+ \frac{1}{2} \frac{\partial^2 I(\vec{x},\vec{n}^\prime,\nu)}{\partial \nu^2 }
\langle \Delta \nu^2 \rangle
\right]
\ee 
To perform the integrals, the Eddington approximation for the angular dependence is inserted with the following result
\be
j_{\rm sc} & = & k\phi J - k\phi \Delta \langle u_\parallel \rangle \left( \frac{\partial J}{\partial \nu} - \frac{6}{5} \vec{n} \cdot \frac{\partial \vec{H}}{\partial \nu} \right)
+ \frac{1}{2} \Delta^2 k\phi \left[ 
\frac{\partial^2 J}{\partial \nu^2} \left( \frac{7}{5} \langle u_\parallel^2 \rangle + \frac{3}{10} \right)
- \frac{12}{5} \langle u_\parallel^2 \rangle 
\vec{n} \cdot \frac{\partial^2 \vec{H}}{\partial \nu^2} \right].
\label{eq:jsc}
\ee
The first term in Equation \ref{eq:jsc}, $k\phi J$, represents re-emission of the photon through de-excitation of the atom. It cancels the $-k\phi J$ term in Equation \ref{eq:rteqn} that corresponds to excitation of the atom. The third and fifth terms in Equation \ref{eq:jsc} involve frequency derivatives of $\vec{H}$. If carried through the calculation, they end up giving terms smaller than then largest terms by a factor of $1/x^2$, which is small on the line wing. These terms are ignored from here on.

This far the transfer equation is
\be
\vec{n} \cdot \grad \left( J + 3 \vec{n} \cdot \vec{H} \right)
& =& j_{\rm em}
- \left( k\phi + \alpha_{\rm abs} \right) \left( J + 3 \vec{n} \cdot \vec{H} \right)
\nonumber \\ & + & 
(1-p) \left[ k\phi J - k\phi \Delta \langle u_\parallel \rangle \frac{\partial J}{\partial \nu} 
+ \frac{1}{2} \Delta^2 k\phi 
\frac{\partial^2 J}{\partial \nu^2} \left( \frac{7}{5} \langle u_\parallel^2 \rangle + \frac{3}{10} \right)
\right].
\label{eq:rteqn2}
\ee
The moment equations are
\be
\grad \cdot \vec{H} & = & j_{\rm em} 
- \left( \alpha_{\rm abs} + pk\phi\right) J
- k\phi \Delta \langle u_\parallel \rangle \frac{\partial J}{\partial \nu} 
+ \frac{1}{2} \Delta^2 k\phi 
\frac{\partial^2 J}{\partial \nu^2} \left( \frac{7}{5} \langle u_\parallel^2 \rangle + \frac{3}{10} \right)
\ee
and
\be
\frac{1}{3} \grad J & =& - \left( k \phi + \alpha_{\rm abs} \right) \vec{H}.
\ee
Assuming the coefficients are constant in space, these two equations can be combined together to find
\be
- \frac{1}{3 (k\phi + \alpha_{\rm abs})} \nabla^2 J
& =& j_{\rm em} 
- \left( \alpha_{\rm abs} + pk\phi\right) J
- k\phi \Delta \langle u_\parallel \rangle \frac{\partial J}{\partial \nu} 
+ \frac{1}{2} \Delta^2 k\phi 
\frac{\partial^2 J}{\partial \nu^2} \left( \frac{7}{5} \langle u_\parallel^2 \rangle + \frac{3}{10} \right)
\ee



\bibliography{ref.bib}{}
\bibliographystyle{aasjournal}



\end{document}

